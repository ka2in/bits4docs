%% Generated by Sphinx.
\def\sphinxdocclass{report}
\documentclass[a4paper,10pt,english,openany,oneside]{sphinxmanual}
\ifdefined\pdfpxdimen
   \let\sphinxpxdimen\pdfpxdimen\else\newdimen\sphinxpxdimen
\fi \sphinxpxdimen=.75bp\relax
\ifdefined\pdfimageresolution
    \pdfimageresolution= \numexpr \dimexpr1in\relax/\sphinxpxdimen\relax
\fi
%% let collapsible pdf bookmarks panel have high depth per default
\PassOptionsToPackage{bookmarksdepth=5}{hyperref}
%% turn off hyperref patch of \index as sphinx.xdy xindy module takes care of
%% suitable \hyperpage mark-up, working around hyperref-xindy incompatibility
\PassOptionsToPackage{hyperindex=false}{hyperref}
%% memoir class requires extra handling
\makeatletter\@ifclassloaded{memoir}
{\ifdefined\memhyperindexfalse\memhyperindexfalse\fi}{}\makeatother

\PassOptionsToPackage{warn}{textcomp}

\catcode`^^^^00a0\active\protected\def^^^^00a0{\leavevmode\nobreak\ }
\usepackage{cmap}
\usepackage{fontspec}
\defaultfontfeatures[\rmfamily,\sffamily,\ttfamily]{}
\usepackage{amsmath,amssymb,amstext}
\usepackage{polyglossia}
\setmainlanguage{english}



\setmainfont{FreeSerif}[
  Extension      = .otf,
  UprightFont    = *,
  ItalicFont     = *Italic,
  BoldFont       = *Bold,
  BoldItalicFont = *BoldItalic
]
\setsansfont{FreeSans}[
  Extension      = .otf,
  UprightFont    = *,
  ItalicFont     = *Oblique,
  BoldFont       = *Bold,
  BoldItalicFont = *BoldOblique,
]
\setmonofont{FreeMono}[
  Extension      = .otf,
  UprightFont    = *,
  ItalicFont     = *Oblique,
  BoldFont       = *Bold,
  BoldItalicFont = *BoldOblique,
]



\usepackage[Bjornstrup]{fncychap}
\usepackage{sphinx}

\fvset{fontsize=\small}
\usepackage{geometry}


% Include hyperref last.
\usepackage{hyperref}
% Fix anchor placement for figures with captions.
\usepackage{hypcap}% it must be loaded after hyperref.
% Set up styles of URL: it should be placed after hyperref.
\urlstyle{same}

\addto\captionsenglish{\renewcommand{\contentsname}{Table of Contents}}

\usepackage{sphinxmessages}
\setcounter{tocdepth}{3}
\setcounter{secnumdepth}{3}
\usepackage[bottom]{footmisc}

\title{bits4docs Documentation}
\date{Jun 05, 2022}
\release{1.0}
\author{GlobalTech Translations}
\newcommand{\sphinxlogo}{\vbox{}}
\renewcommand{\releasename}{Release}
\makeindex
\begin{document}

\pagestyle{empty}
\sphinxmaketitle
\pagestyle{plain}
\sphinxtableofcontents
\pagestyle{normal}
\phantomsection\label{\detokenize{index::doc}}


\sphinxAtStartPar
You will find here articles, tutorials and other resources about the following topics:
\begin{itemize}
\item {} 
\sphinxAtStartPar
Technical communication

\item {} 
\sphinxAtStartPar
Software documentation

\item {} 
\sphinxAtStartPar
Front/backend web development

\end{itemize}

\sphinxAtStartPar
\sphinxstylestrong{About the author:}

\sphinxAtStartPar
Fayçal Alami\sphinxhyphen{}Hassani \sphinxhyphen{} \sphinxhref{https://globaltech-translations.com}{@GlobalTech Translations}%
\begin{footnote}[1]\sphinxAtStartFootnote
\sphinxnolinkurl{https://globaltech-translations.com}
%
\end{footnote} \sphinxhyphen{} \sphinxhref{https://fosstodon.org/@gnufcl}{@gnufcl@fosstodon.org}%
\begin{footnote}[2]\sphinxAtStartFootnote
\sphinxnolinkurl{https://fosstodon.org/@gnufcl}
%
\end{footnote}
\begin{itemize}
\item {} 
\sphinxAtStartPar
Technical communicator, translator and interpreter

\item {} 
\sphinxAtStartPar
Markup: reStructuredText, Markdown, DocBook, XML

\item {} 
\sphinxAtStartPar
Web development: HTML, CSS, JavaScript, jQuery, MySQL

\item {} 
\sphinxAtStartPar
Text editors: Nano, Atom, Sublime Text

\item {} 
\sphinxAtStartPar
Version control: Git, CVS

\item {} 
\sphinxAtStartPar
OS: Debian, Fedora

\item {} 
\sphinxAtStartPar
Now learning: Docs\sphinxhyphen{}as\sphinxhyphen{}Code based on \sphinxhref{https://antora.org/}{Antora}%
\begin{footnote}[3]\sphinxAtStartFootnote
\sphinxnolinkurl{https://antora.org/}
%
\end{footnote} with \sphinxhref{https://asciidoc-py.github.io/}{AsciiDoc}%
\begin{footnote}[4]\sphinxAtStartFootnote
\sphinxnolinkurl{https://asciidoc-py.github.io/}
%
\end{footnote}

\item {} 
\sphinxAtStartPar
Currently reading:
\begin{quote}

\sphinxAtStartPar
📕 Web Security for Developers \sphinxhyphen{} Real Threats, Practical Defense \sphinxhyphen{} Malcolm McDonald \sphinxhyphen{} No Starch Press \sphinxhyphen{} ISBN: 978\sphinxhyphen{}1\sphinxhyphen{}59327\sphinxhyphen{}994\sphinxhyphen{}3

\sphinxAtStartPar
📕 Eloquent JavaScript \sphinxhyphen{} A Modern Introduction to Programming \sphinxhyphen{} Marijn Haverbeke \sphinxhyphen{} No Starch Press \sphinxhyphen{} 978\sphinxhyphen{}1\sphinxhyphen{}59327950\sphinxhyphen{}9
\end{quote}

\end{itemize}

\begin{sphinxadmonition}{note}{Note:}
\sphinxAtStartPar
This project is under active development. If you have any questions, please send an email to: info{[}@{]}globaltech\sphinxhyphen{}translations{[}.{]}com \sphinxhyphen{} \sphinxcode{\sphinxupquote{PGP KeyID: 0x52D6AF10}}
\end{sphinxadmonition}

\sphinxstepscope


\chapter{FTP vs. API – differences in terms of data transmission}
\label{\detokenize{ftp-vs-api:ftp-vs-api-differences-in-terms-of-data-transmission}}\label{\detokenize{ftp-vs-api::doc}}
\sphinxAtStartPar
Users can choose either to send their data to an FTP server or via an API. These two connectivity options have different implications in terms of security, access possibilities, and customer experience.

\begin{figure}[H]
\centering

\noindent\sphinxincludegraphics[width=527.12116\sphinxpxdimen,height=346.62805\sphinxpxdimen]{{network-ftp}.png}
\end{figure}


\chapter{FTP – an old, well\sphinxhyphen{}established protocol}
\label{\detokenize{ftp-vs-api:ftp-an-old-well-established-protocol}}
\sphinxAtStartPar
FTP (\sphinxstyleemphasis{File Transfer Protocol}) uses a client/server model to allow users to move files between a local machine (\sphinxstyleemphasis{client}) and a remote host (\sphinxstyleemphasis{server}). FTP is an easy and convenient method to download and upload large data volumes.

\sphinxAtStartPar
One particular aspect of FTP is that it relies on “two” logical TCP connections to ensure communication between the client and the server:
\begin{itemize}
\item {} 
\sphinxAtStartPar
\sphinxstylestrong{Control connection:} This primary communication channel ensures the transmission of control traffic over port 21 and remains active during the entire FTP session. Control traffic includes FTP commands and replies.

\item {} 
\sphinxAtStartPar
\sphinxstylestrong{Data connection:} Whenever you need to transfer files between a client and a remote server or vice versa, FTP will initiate this TCP connection to ensure data transmission over port 20. Unlike the control connection, a data connection does not remain active during the entire FTP session and ends immediately after the file transfer.

\end{itemize}

\sphinxAtStartPar
FTP uses a simple authentication mechanism that consists in using a “user name” and a “password”. The client sends the authentication data to the remote server using the FTP commands: \sphinxcode{\sphinxupquote{USER}} and \sphinxcode{\sphinxupquote{PASS}}.

\sphinxAtStartPar
The FTP standard defines three main categories of FTP commands:
\begin{itemize}
\item {} 
\sphinxAtStartPar
\sphinxcode{\sphinxupquote{Access Control Commands}}

\item {} 
\sphinxAtStartPar
\sphinxcode{\sphinxupquote{Transfer Parameter Commands}}

\item {} 
\sphinxAtStartPar
\sphinxcode{\sphinxupquote{FTP Service Commands}}

\end{itemize}

\sphinxAtStartPar
The following tables provide an overview of the different commands within each category:


\begin{savenotes}\sphinxattablestart
\centering
\begin{tabulary}{\linewidth}[t]{|T|T|T|T|}
\hline
\sphinxstartmulticolumn{4}%
\begin{varwidth}[t]{\sphinxcolwidth{4}{4}}
\sphinxstyletheadfamily \sphinxAtStartPar
\sphinxstylestrong{Access Control Commands}
\par
\vskip-\baselineskip\vbox{\hbox{\strut}}\end{varwidth}%
\sphinxstopmulticolumn
\\
\hline
\sphinxAtStartPar
\sphinxcode{\sphinxupquote{USER}}
&
\sphinxAtStartPar
\sphinxstyleemphasis{User Name}
&\sphinxstartmulticolumn{2}%
\begin{varwidth}[t]{\sphinxcolwidth{2}{4}}
\sphinxAtStartPar
User identification to access the server’s file system
\par
\vskip-\baselineskip\vbox{\hbox{\strut}}\end{varwidth}%
\sphinxstopmulticolumn
\\
\hline
\sphinxAtStartPar
\sphinxcode{\sphinxupquote{PASS}}
&
\sphinxAtStartPar
\sphinxstyleemphasis{Password}
&\sphinxstartmulticolumn{2}%
\begin{varwidth}[t]{\sphinxcolwidth{2}{4}}
\sphinxAtStartPar
Command that follows the \sphinxcode{\sphinxupquote{USER}} command immediately
\par
\vskip-\baselineskip\vbox{\hbox{\strut}}\end{varwidth}%
\sphinxstopmulticolumn
\\
\hline
\sphinxAtStartPar
\sphinxcode{\sphinxupquote{ACCT}}
&
\sphinxAtStartPar
\sphinxstyleemphasis{Account}
&\sphinxstartmulticolumn{2}%
\begin{varwidth}[t]{\sphinxcolwidth{2}{4}}
\sphinxAtStartPar
For login purposes or tasks requiring specific access
\par
\vskip-\baselineskip\vbox{\hbox{\strut}}\end{varwidth}%
\sphinxstopmulticolumn
\\
\hline
\sphinxAtStartPar
\sphinxcode{\sphinxupquote{CWD}}
&
\sphinxAtStartPar
\sphinxstyleemphasis{Change Working}
\sphinxstyleemphasis{Directory}
&\sphinxstartmulticolumn{2}%
\begin{varwidth}[t]{\sphinxcolwidth{2}{4}}
\sphinxAtStartPar
Store or retrieve files on a different directory
without modifying login or account information
\par
\vskip-\baselineskip\vbox{\hbox{\strut}}\end{varwidth}%
\sphinxstopmulticolumn
\\
\hline
\sphinxAtStartPar
\sphinxcode{\sphinxupquote{CDUP}}
&
\sphinxAtStartPar
\sphinxstyleemphasis{Change}
\sphinxstyleemphasis{Directory Up}
&\sphinxstartmulticolumn{2}%
\begin{varwidth}[t]{\sphinxcolwidth{2}{4}}
\sphinxAtStartPar
Transfer directory trees between operating systems
that use different syntaxes to name the parent
directory
\par
\vskip-\baselineskip\vbox{\hbox{\strut}}\end{varwidth}%
\sphinxstopmulticolumn
\\
\hline
\sphinxAtStartPar
\sphinxcode{\sphinxupquote{SMNT}}
&
\sphinxAtStartPar
\sphinxstyleemphasis{Structure Mount}
&\sphinxstartmulticolumn{2}%
\begin{varwidth}[t]{\sphinxcolwidth{2}{4}}
\sphinxAtStartPar
Mount a different file system data structure without
modifying the login or accounting information
\par
\vskip-\baselineskip\vbox{\hbox{\strut}}\end{varwidth}%
\sphinxstopmulticolumn
\\
\hline
\sphinxAtStartPar
\sphinxcode{\sphinxupquote{REIN}}
&
\sphinxAtStartPar
\sphinxstyleemphasis{Reinitialize}
&\sphinxstartmulticolumn{2}%
\begin{varwidth}[t]{\sphinxcolwidth{2}{4}}
\sphinxAtStartPar
Reset parameters to the default settings and flush
account information and all Input/Output
\par
\vskip-\baselineskip\vbox{\hbox{\strut}}\end{varwidth}%
\sphinxstopmulticolumn
\\
\hline
\sphinxAtStartPar
\sphinxcode{\sphinxupquote{QUIT}}
&
\sphinxAtStartPar
\sphinxstyleemphasis{Logout}
&\sphinxstartmulticolumn{2}%
\begin{varwidth}[t]{\sphinxcolwidth{2}{4}}
\sphinxAtStartPar
Terminate USER session and close control connection
\par
\vskip-\baselineskip\vbox{\hbox{\strut}}\end{varwidth}%
\sphinxstopmulticolumn
\\
\hline
\end{tabulary}
\par
\sphinxattableend\end{savenotes}


\begin{savenotes}\sphinxattablestart
\centering
\begin{tabulary}{\linewidth}[t]{|T|T|T|T|T|T|T|}
\hline
\sphinxstartmulticolumn{7}%
\begin{varwidth}[t]{\sphinxcolwidth{7}{7}}
\sphinxstyletheadfamily \sphinxAtStartPar
\sphinxstylestrong{Transfer Parameter Commands}
\par
\vskip-\baselineskip\vbox{\hbox{\strut}}\end{varwidth}%
\sphinxstopmulticolumn
\\
\hline
\sphinxAtStartPar
\sphinxcode{\sphinxupquote{PORT}}
&\sphinxstartmulticolumn{2}%
\begin{varwidth}[t]{\sphinxcolwidth{2}{7}}
\sphinxAtStartPar
\sphinxstyleemphasis{Data Port}
\par
\vskip-\baselineskip\vbox{\hbox{\strut}}\end{varwidth}%
\sphinxstopmulticolumn
&\sphinxstartmulticolumn{4}%
\begin{varwidth}[t]{\sphinxcolwidth{4}{7}}
\sphinxAtStartPar
Specify port number to use for data connection
\par
\vskip-\baselineskip\vbox{\hbox{\strut}}\end{varwidth}%
\sphinxstopmulticolumn
\\
\hline
\sphinxAtStartPar
\sphinxcode{\sphinxupquote{PASV}}
&\sphinxstartmulticolumn{2}%
\begin{varwidth}[t]{\sphinxcolwidth{2}{7}}
\sphinxAtStartPar
\sphinxstyleemphasis{Passive}
\par
\vskip-\baselineskip\vbox{\hbox{\strut}}\end{varwidth}%
\sphinxstopmulticolumn
&\sphinxstartmulticolumn{4}%
\begin{varwidth}[t]{\sphinxcolwidth{4}{7}}
\sphinxAtStartPar
Request the Server Data Transfer Process to
\sphinxstyleemphasis{listen} on a non\sphinxhyphen{}default data port
\par
\vskip-\baselineskip\vbox{\hbox{\strut}}\end{varwidth}%
\sphinxstopmulticolumn
\\
\hline
\sphinxAtStartPar
\sphinxcode{\sphinxupquote{TYPE}}
&\sphinxstartmulticolumn{2}%
\begin{varwidth}[t]{\sphinxcolwidth{2}{7}}
\sphinxAtStartPar
\sphinxstyleemphasis{Representation}
\sphinxstyleemphasis{Type}
\par
\vskip-\baselineskip\vbox{\hbox{\strut}}\end{varwidth}%
\sphinxstopmulticolumn
&\sphinxstartmulticolumn{4}%
\begin{varwidth}[t]{\sphinxcolwidth{4}{7}}
\sphinxAtStartPar
Inform the server about the data type of
files that are transferred by the client
\par
\vskip-\baselineskip\vbox{\hbox{\strut}}\end{varwidth}%
\sphinxstopmulticolumn
\\
\hline
\sphinxAtStartPar
\sphinxcode{\sphinxupquote{STRU}}
&\sphinxstartmulticolumn{2}%
\begin{varwidth}[t]{\sphinxcolwidth{2}{7}}
\sphinxAtStartPar
\sphinxstyleemphasis{File Structure}
\par
\vskip-\baselineskip\vbox{\hbox{\strut}}\end{varwidth}%
\sphinxstopmulticolumn
&\sphinxstartmulticolumn{4}%
\begin{varwidth}[t]{\sphinxcolwidth{4}{7}}
\sphinxAtStartPar
Specify the data structure for the file
(\sphinxstylestrong{File}, \sphinxstylestrong{Record}, or \sphinxstylestrong{Page})
\par
\vskip-\baselineskip\vbox{\hbox{\strut}}\end{varwidth}%
\sphinxstopmulticolumn
\\
\hline
\sphinxAtStartPar
\sphinxcode{\sphinxupquote{MODE}}
&\sphinxstartmulticolumn{2}%
\begin{varwidth}[t]{\sphinxcolwidth{2}{7}}
\sphinxAtStartPar
\sphinxstyleemphasis{Transfer Mode}
\par
\vskip-\baselineskip\vbox{\hbox{\strut}}\end{varwidth}%
\sphinxstopmulticolumn
&\sphinxstartmulticolumn{4}%
\begin{varwidth}[t]{\sphinxcolwidth{4}{7}}
\sphinxAtStartPar
Specify the transmission mode to use
(\sphinxstylestrong{Stream}, \sphinxstylestrong{Block}, or \sphinxstylestrong{Compressed})
\par
\vskip-\baselineskip\vbox{\hbox{\strut}}\end{varwidth}%
\sphinxstopmulticolumn
\\
\hline
\end{tabulary}
\par
\sphinxattableend\end{savenotes}


\begin{savenotes}\sphinxattablestart
\centering
\begin{tabulary}{\linewidth}[t]{|T|T|T|T|T|T|T|}
\hline
\sphinxstartmulticolumn{7}%
\begin{varwidth}[t]{\sphinxcolwidth{7}{7}}
\sphinxstyletheadfamily \sphinxAtStartPar
\sphinxstylestrong{FTP Service Commands}
\par
\vskip-\baselineskip\vbox{\hbox{\strut}}\end{varwidth}%
\sphinxstopmulticolumn
\\
\hline
\sphinxAtStartPar
\sphinxcode{\sphinxupquote{RETR}}
&\sphinxstartmulticolumn{2}%
\begin{varwidth}[t]{\sphinxcolwidth{2}{7}}
\sphinxAtStartPar
\sphinxstyleemphasis{Retrieve}
\par
\vskip-\baselineskip\vbox{\hbox{\strut}}\end{varwidth}%
\sphinxstopmulticolumn
&\sphinxstartmulticolumn{4}%
\begin{varwidth}[t]{\sphinxcolwidth{4}{7}}
\sphinxAtStartPar
Transfer a file from the server to the client
\par
\vskip-\baselineskip\vbox{\hbox{\strut}}\end{varwidth}%
\sphinxstopmulticolumn
\\
\hline
\sphinxAtStartPar
\sphinxcode{\sphinxupquote{STOR}}
&\sphinxstartmulticolumn{2}%
\begin{varwidth}[t]{\sphinxcolwidth{2}{7}}
\sphinxAtStartPar
\sphinxstyleemphasis{Store}
\par
\vskip-\baselineskip\vbox{\hbox{\strut}}\end{varwidth}%
\sphinxstopmulticolumn
&\sphinxstartmulticolumn{4}%
\begin{varwidth}[t]{\sphinxcolwidth{4}{7}}
\sphinxAtStartPar
Store data as a file on the server
\par
\vskip-\baselineskip\vbox{\hbox{\strut}}\end{varwidth}%
\sphinxstopmulticolumn
\\
\hline
\sphinxAtStartPar
\sphinxcode{\sphinxupquote{STOU}}
&\sphinxstartmulticolumn{2}%
\begin{varwidth}[t]{\sphinxcolwidth{2}{7}}
\sphinxAtStartPar
\sphinxstyleemphasis{Store Unique}
\par
\vskip-\baselineskip\vbox{\hbox{\strut}}\end{varwidth}%
\sphinxstopmulticolumn
&\sphinxstartmulticolumn{4}%
\begin{varwidth}[t]{\sphinxcolwidth{4}{7}}
\sphinxAtStartPar
Similar to \sphinxcode{\sphinxupquote{STOR}}, but the file must have a
unique name inside the current directory
\par
\vskip-\baselineskip\vbox{\hbox{\strut}}\end{varwidth}%
\sphinxstopmulticolumn
\\
\hline
\sphinxAtStartPar
\sphinxcode{\sphinxupquote{APPE}}
&\sphinxstartmulticolumn{2}%
\begin{varwidth}[t]{\sphinxcolwidth{2}{7}}
\sphinxAtStartPar
\sphinxstyleemphasis{Append}
\par
\vskip-\baselineskip\vbox{\hbox{\strut}}\end{varwidth}%
\sphinxstopmulticolumn
&\sphinxstartmulticolumn{4}%
\begin{varwidth}[t]{\sphinxcolwidth{4}{7}}
\sphinxAtStartPar
If a file with same name already exists on the
server, the data is appended to the existing file
\par
\vskip-\baselineskip\vbox{\hbox{\strut}}\end{varwidth}%
\sphinxstopmulticolumn
\\
\hline
\sphinxAtStartPar
\sphinxcode{\sphinxupquote{ALLO}}
&\sphinxstartmulticolumn{2}%
\begin{varwidth}[t]{\sphinxcolwidth{2}{7}}
\sphinxAtStartPar
\sphinxstyleemphasis{Allocate}
\par
\vskip-\baselineskip\vbox{\hbox{\strut}}\end{varwidth}%
\sphinxstopmulticolumn
&\sphinxstartmulticolumn{4}%
\begin{varwidth}[t]{\sphinxcolwidth{4}{7}}
\sphinxAtStartPar
Make sure that sufficient storage is available on
the server before data transmission
\par
\vskip-\baselineskip\vbox{\hbox{\strut}}\end{varwidth}%
\sphinxstopmulticolumn
\\
\hline
\sphinxAtStartPar
\sphinxcode{\sphinxupquote{REST}}
&\sphinxstartmulticolumn{2}%
\begin{varwidth}[t]{\sphinxcolwidth{2}{7}}
\sphinxAtStartPar
\sphinxstyleemphasis{Restart}
\par
\vskip-\baselineskip\vbox{\hbox{\strut}}\end{varwidth}%
\sphinxstopmulticolumn
&\sphinxstartmulticolumn{4}%
\begin{varwidth}[t]{\sphinxcolwidth{4}{7}}
\sphinxAtStartPar
Restart file transfer at a specific server marker
\par
\vskip-\baselineskip\vbox{\hbox{\strut}}\end{varwidth}%
\sphinxstopmulticolumn
\\
\hline
\sphinxAtStartPar
\sphinxcode{\sphinxupquote{RNFR}}
&\sphinxstartmulticolumn{2}%
\begin{varwidth}[t]{\sphinxcolwidth{2}{7}}
\sphinxAtStartPar
\sphinxstyleemphasis{Rename From}
\par
\vskip-\baselineskip\vbox{\hbox{\strut}}\end{varwidth}%
\sphinxstopmulticolumn
&\sphinxstartmulticolumn{4}%
\begin{varwidth}[t]{\sphinxcolwidth{4}{7}}
\sphinxAtStartPar
Specify the old name of the file to be renamed.
Must be followed by the \sphinxcode{\sphinxupquote{RNTO}} command
\par
\vskip-\baselineskip\vbox{\hbox{\strut}}\end{varwidth}%
\sphinxstopmulticolumn
\\
\hline
\sphinxAtStartPar
\sphinxcode{\sphinxupquote{RNTO}}
&\sphinxstartmulticolumn{2}%
\begin{varwidth}[t]{\sphinxcolwidth{2}{7}}
\sphinxAtStartPar
\sphinxstyleemphasis{Rename To}
\par
\vskip-\baselineskip\vbox{\hbox{\strut}}\end{varwidth}%
\sphinxstopmulticolumn
&\sphinxstartmulticolumn{4}%
\begin{varwidth}[t]{\sphinxcolwidth{4}{7}}
\sphinxAtStartPar
Specify the new name of the file to be renamed.
\par
\vskip-\baselineskip\vbox{\hbox{\strut}}\end{varwidth}%
\sphinxstopmulticolumn
\\
\hline
\sphinxAtStartPar
\sphinxcode{\sphinxupquote{ABOR}}
&\sphinxstartmulticolumn{2}%
\begin{varwidth}[t]{\sphinxcolwidth{2}{7}}
\sphinxAtStartPar
\sphinxstyleemphasis{Abort}
\par
\vskip-\baselineskip\vbox{\hbox{\strut}}\end{varwidth}%
\sphinxstopmulticolumn
&\sphinxstartmulticolumn{4}%
\begin{varwidth}[t]{\sphinxcolwidth{4}{7}}
\sphinxAtStartPar
Instruct the server to abort the last FTP command
and any associated data transfer
\par
\vskip-\baselineskip\vbox{\hbox{\strut}}\end{varwidth}%
\sphinxstopmulticolumn
\\
\hline
\sphinxAtStartPar
\sphinxcode{\sphinxupquote{DELE}}
&\sphinxstartmulticolumn{2}%
\begin{varwidth}[t]{\sphinxcolwidth{2}{7}}
\sphinxAtStartPar
\sphinxstyleemphasis{Delete}
\par
\vskip-\baselineskip\vbox{\hbox{\strut}}\end{varwidth}%
\sphinxstopmulticolumn
&\sphinxstartmulticolumn{4}%
\begin{varwidth}[t]{\sphinxcolwidth{4}{7}}
\sphinxAtStartPar
Remove the specified file from the server
\par
\vskip-\baselineskip\vbox{\hbox{\strut}}\end{varwidth}%
\sphinxstopmulticolumn
\\
\hline
\sphinxAtStartPar
\sphinxcode{\sphinxupquote{RMD}}
&\sphinxstartmulticolumn{2}%
\begin{varwidth}[t]{\sphinxcolwidth{2}{7}}
\sphinxAtStartPar
\sphinxstyleemphasis{Remove}
\sphinxstyleemphasis{Directory}
\par
\vskip-\baselineskip\vbox{\hbox{\strut}}\end{varwidth}%
\sphinxstopmulticolumn
&\sphinxstartmulticolumn{4}%
\begin{varwidth}[t]{\sphinxcolwidth{4}{7}}
\sphinxAtStartPar
Remove the specified directory from the server
\par
\vskip-\baselineskip\vbox{\hbox{\strut}}\end{varwidth}%
\sphinxstopmulticolumn
\\
\hline
\sphinxAtStartPar
\sphinxcode{\sphinxupquote{MKD}}
&\sphinxstartmulticolumn{2}%
\begin{varwidth}[t]{\sphinxcolwidth{2}{7}}
\sphinxAtStartPar
\sphinxstyleemphasis{Make Directory}
\par
\vskip-\baselineskip\vbox{\hbox{\strut}}\end{varwidth}%
\sphinxstopmulticolumn
&\sphinxstartmulticolumn{4}%
\begin{varwidth}[t]{\sphinxcolwidth{4}{7}}
\sphinxAtStartPar
Create a directory
\par
\vskip-\baselineskip\vbox{\hbox{\strut}}\end{varwidth}%
\sphinxstopmulticolumn
\\
\hline
\sphinxAtStartPar
\sphinxcode{\sphinxupquote{PWD}}
&\sphinxstartmulticolumn{2}%
\begin{varwidth}[t]{\sphinxcolwidth{2}{7}}
\sphinxAtStartPar
\sphinxstyleemphasis{Print Working}
\sphinxstyleemphasis{Directory}
\par
\vskip-\baselineskip\vbox{\hbox{\strut}}\end{varwidth}%
\sphinxstopmulticolumn
&\sphinxstartmulticolumn{4}%
\begin{varwidth}[t]{\sphinxcolwidth{4}{7}}
\sphinxAtStartPar
Display the current working directory on the
server
\par
\vskip-\baselineskip\vbox{\hbox{\strut}}\end{varwidth}%
\sphinxstopmulticolumn
\\
\hline
\sphinxAtStartPar
\sphinxcode{\sphinxupquote{LIST}}
&\sphinxstartmulticolumn{2}%
\begin{varwidth}[t]{\sphinxcolwidth{2}{7}}
\sphinxAtStartPar
\sphinxstyleemphasis{List}
\par
\vskip-\baselineskip\vbox{\hbox{\strut}}\end{varwidth}%
\sphinxstopmulticolumn
&\sphinxstartmulticolumn{4}%
\begin{varwidth}[t]{\sphinxcolwidth{4}{7}}
\sphinxAtStartPar
Instruct the server to send a list of the content
available in the current directory
\par
\vskip-\baselineskip\vbox{\hbox{\strut}}\end{varwidth}%
\sphinxstopmulticolumn
\\
\hline
\sphinxAtStartPar
\sphinxcode{\sphinxupquote{NLST}}
&\sphinxstartmulticolumn{2}%
\begin{varwidth}[t]{\sphinxcolwidth{2}{7}}
\sphinxAtStartPar
\sphinxstyleemphasis{Name List}
\par
\vskip-\baselineskip\vbox{\hbox{\strut}}\end{varwidth}%
\sphinxstopmulticolumn
&\sphinxstartmulticolumn{4}%
\begin{varwidth}[t]{\sphinxcolwidth{4}{7}}
\sphinxAtStartPar
Similar to \sphinxcode{\sphinxupquote{LIST}}, but only sends a directory
listing
\par
\vskip-\baselineskip\vbox{\hbox{\strut}}\end{varwidth}%
\sphinxstopmulticolumn
\\
\hline
\sphinxAtStartPar
\sphinxcode{\sphinxupquote{SITE}}
&\sphinxstartmulticolumn{2}%
\begin{varwidth}[t]{\sphinxcolwidth{2}{7}}
\sphinxAtStartPar
\sphinxstyleemphasis{Site}
\sphinxstyleemphasis{Parameters}
\par
\vskip-\baselineskip\vbox{\hbox{\strut}}\end{varwidth}%
\sphinxstopmulticolumn
&\sphinxstartmulticolumn{4}%
\begin{varwidth}[t]{\sphinxcolwidth{4}{7}}
\sphinxAtStartPar
Server\sphinxhyphen{}side commands to use specific functions
that are required for data transfer
\par
\vskip-\baselineskip\vbox{\hbox{\strut}}\end{varwidth}%
\sphinxstopmulticolumn
\\
\hline
\sphinxAtStartPar
\sphinxcode{\sphinxupquote{SYST}}
&\sphinxstartmulticolumn{2}%
\begin{varwidth}[t]{\sphinxcolwidth{2}{7}}
\sphinxAtStartPar
\sphinxstyleemphasis{System}
\par
\vskip-\baselineskip\vbox{\hbox{\strut}}\end{varwidth}%
\sphinxstopmulticolumn
&\sphinxstartmulticolumn{4}%
\begin{varwidth}[t]{\sphinxcolwidth{4}{7}}
\sphinxAtStartPar
Instruct the server to send information about its
operating system
\par
\vskip-\baselineskip\vbox{\hbox{\strut}}\end{varwidth}%
\sphinxstopmulticolumn
\\
\hline
\sphinxAtStartPar
\sphinxcode{\sphinxupquote{STAT}}
&\sphinxstartmulticolumn{2}%
\begin{varwidth}[t]{\sphinxcolwidth{2}{7}}
\sphinxAtStartPar
\sphinxstyleemphasis{Status}
\par
\vskip-\baselineskip\vbox{\hbox{\strut}}\end{varwidth}%
\sphinxstopmulticolumn
&\sphinxstartmulticolumn{4}%
\begin{varwidth}[t]{\sphinxcolwidth{4}{7}}
\sphinxAtStartPar
Instruct the server to indicate the status of a
file or the ongoing data transfer
\par
\vskip-\baselineskip\vbox{\hbox{\strut}}\end{varwidth}%
\sphinxstopmulticolumn
\\
\hline
\sphinxAtStartPar
\sphinxcode{\sphinxupquote{HELP}}
&\sphinxstartmulticolumn{2}%
\begin{varwidth}[t]{\sphinxcolwidth{2}{7}}
\sphinxAtStartPar
\sphinxstyleemphasis{Help}
\par
\vskip-\baselineskip\vbox{\hbox{\strut}}\end{varwidth}%
\sphinxstopmulticolumn
&\sphinxstartmulticolumn{4}%
\begin{varwidth}[t]{\sphinxcolwidth{4}{7}}
\sphinxAtStartPar
Prompt the server to send help information that
shows how to use the server
\par
\vskip-\baselineskip\vbox{\hbox{\strut}}\end{varwidth}%
\sphinxstopmulticolumn
\\
\hline
\sphinxAtStartPar
\sphinxcode{\sphinxupquote{NOOP}}
&\sphinxstartmulticolumn{2}%
\begin{varwidth}[t]{\sphinxcolwidth{2}{7}}
\sphinxAtStartPar
\sphinxstyleemphasis{No Operation}
\par
\vskip-\baselineskip\vbox{\hbox{\strut}}\end{varwidth}%
\sphinxstopmulticolumn
&\sphinxstartmulticolumn{4}%
\begin{varwidth}[t]{\sphinxcolwidth{4}{7}}
\sphinxAtStartPar
Prompt the server to send an OK reply but does
not impact the previously entered commands
\par
\vskip-\baselineskip\vbox{\hbox{\strut}}\end{varwidth}%
\sphinxstopmulticolumn
\\
\hline
\end{tabulary}
\par
\sphinxattableend\end{savenotes}

\sphinxAtStartPar
For a more detailed description, please refer to The FTP specification \sphinxhref{https://www.w3.org/Protocols/rfc959/4\_FileTransfer.html}{RFC959}%
\begin{footnote}[5]\sphinxAtStartFootnote
\sphinxnolinkurl{https://www.w3.org/Protocols/rfc959/4\_FileTransfer.html}
%
\end{footnote}.


\chapter{FTP and security}
\label{\detokenize{ftp-vs-api:ftp-and-security}}
\sphinxAtStartPar
Data transmission with the basic FTP protocol is insecure because it is unencrypted. For a secure data transfer, you need to use FTPS (\sphinxstyleemphasis{FTP over SSL}) or SFTP (\sphinxstyleemphasis{SSH File Transfer Protocol}). Unlike FTPS, which requires opening multiple ports for data transmission, SFTP only needs a single port number to transfer the data. Therefore, SFTP is more suitable for firewall security.

\sphinxAtStartPar
While FTP is convenient for large data transfers, its performance in terms of access possibilities and customer experience remains rather limited. For instance, FTP does not allow you to share resources in real\sphinxhyphen{}time between multiple systems, nor does it give you the ability to process data on remote systems.


\chapter{API – more access options for a better customer experience}
\label{\detokenize{ftp-vs-api:api-more-access-options-for-a-better-customer-experience}}
\sphinxAtStartPar
An API (\sphinxstyleemphasis{Application Programming Interface}) is an interface that serves as a bridge between two or more applications. The server\sphinxhyphen{}side components encapsulate the business logic and make it available to multiple clients through the API.

\sphinxAtStartPar
To ensure a secure data transmission, companies can use the HTTPS protocol in conjunction with different encryption methods.
Besides providing real\sphinxhyphen{}time data access to the linked systems, an API integration allows clients to manage and process data by sending requests to the appropriate endpoints.

\sphinxAtStartPar
In the context of HTTP based architectures, clients use \sphinxcode{\sphinxupquote{URIs}} and \sphinxcode{\sphinxupquote{HTTP verbs}} (or methods) to create, request, modify, or delete \sphinxcode{\sphinxupquote{resources}} on a server. A URI (Unique Resource Identifier) allows clients to unequivocally identify a resource that is located on a server. A resource can be anything that is stored on a sever, e.g.:
\begin{itemize}
\item {} 
\sphinxAtStartPar
an employee list in CSV format

\item {} 
\sphinxAtStartPar
a customer database in SQL format

\item {} 
\sphinxAtStartPar
or a presentation file in ODP format

\end{itemize}

\sphinxAtStartPar
The commonly used version of HTTP, i.e. HTTP/1.1, defines eight verbs as the table below shows:


\begin{savenotes}\sphinxattablestart
\centering
\begin{tabular}[t]{|\X{25}{100}|\X{75}{100}|}
\hline

\sphinxAtStartPar
\sphinxstylestrong{HTTP Verb/Method}
&
\sphinxAtStartPar
\sphinxstylestrong{Purpose}
\\
\hline
\sphinxAtStartPar
GET
&
\sphinxAtStartPar
Request a resource
\\
\hline
\sphinxAtStartPar
HEAD
&
\sphinxAtStartPar
Similar to GET, but only provides the HTTP header, and not the entire resource
\\
\hline
\sphinxAtStartPar
POST
&
\sphinxAtStartPar
Generate a resource with a unique ID that is assigned by the server
\\
\hline
\sphinxAtStartPar
PUT
&
\sphinxAtStartPar
Create or replace a resource. The client specifies the resource ID through the URI
\\
\hline
\sphinxAtStartPar
PATCH
&
\sphinxAtStartPar
Partially update a resource that is accessed through its URI
\\
\hline
\sphinxAtStartPar
DELETE
&
\sphinxAtStartPar
Remove a resource that is identified by its URI
\\
\hline
\sphinxAtStartPar
CONNECT
&
\sphinxAtStartPar
Establish an end\sphinxhyphen{}to\sphinxhyphen{}end tunnel connection through a proxy server
\\
\hline
\sphinxAtStartPar
OPTIONS
&
\sphinxAtStartPar
Retrieve information about the available communication options for a given resource
\\
\hline
\end{tabular}
\par
\sphinxattableend\end{savenotes}

\sphinxAtStartPar
APIs offer more advantages over FTP, but they require a higher investment of time and technical expertise.

\sphinxstepscope


\chapter{Git Primer for the Impatient}
\label{\detokenize{gitinminutes:git-primer-for-the-impatient}}\label{\detokenize{gitinminutes::doc}}
\begin{figure}[H]
\centering
\capstart

\noindent\sphinxincludegraphics[width=800\sphinxpxdimen,height=323\sphinxpxdimen]{{gitflow}.pdf}
\caption{Picture under a \sphinxhref{https://creativecommons.org/publicdomain/zero/1.0/}{CC0 1.0 Universal}\sphinxfootnotemark[6] license}\label{\detokenize{gitinminutes:id1}}\end{figure}
%
\begin{footnotetext}[6]\sphinxAtStartFootnote
\sphinxnolinkurl{https://creativecommons.org/publicdomain/zero/1.0/}
%
\end{footnotetext}\ignorespaces 

\section{Short introduction}
\label{\detokenize{gitinminutes:short-introduction}}
\sphinxAtStartPar
Git is a \sphinxstyleabbreviation{VCS} (Version Control System). Basically, a version control system allows you to perform a number of essential tasks, including:
\begin{itemize}
\item {} 
\sphinxAtStartPar
creating a copy of the original project

\item {} 
\sphinxAtStartPar
tracking all the changes that you have made to the project

\item {} 
\sphinxAtStartPar
keeping track of previous versions of the project

\item {} 
\sphinxAtStartPar
marking milestones during the development cycle

\end{itemize}

\sphinxAtStartPar
Git was initially introduced in the Linux community as a revision control system for kernel development. Unlike centralized version control systems such as Subversion and CVS, Git is a fast distributed system.

\sphinxAtStartPar
With Git, you do not need a single central repository to work on your project, since you can work locally on a full clone of the remote repository. What is beautiful about Git is that you can also use it to automate your documentation process.


\section{Git states}
\label{\detokenize{gitinminutes:git-states}}
\sphinxAtStartPar
In a Git workflow, your files will basically go through 3 different states:
\begin{itemize}
\item {} 
\sphinxAtStartPar
Modified: This is when you make changes to the files in your working directory.

\item {} 
\sphinxAtStartPar
Staged: In this intermediate state, Git saves snapshots of the modified files in the staging area.

\item {} 
\sphinxAtStartPar
Committed: Once you commit your changes, Git will save the staged files in the Git directory.

\end{itemize}

\sphinxAtStartPar
The Git directory is a hidden folder \sphinxcode{\sphinxupquote{.git}} at the top level of your working tree.


\section{Installation on Linux}
\label{\detokenize{gitinminutes:installation-on-linux}}
\sphinxAtStartPar
To install Git on Debian based distros, run the following commands:

\begin{sphinxVerbatim}[commandchars=\\\{\}]
\PYG{g+gp}{\PYGZdl{} }sudo apt\PYGZhy{}get update
\PYG{g+gp}{\PYGZdl{} }sudo apt install git\PYGZhy{}all
\end{sphinxVerbatim}

\sphinxAtStartPar
For Red Hat based distros, use the following commands:

\begin{sphinxVerbatim}[commandchars=\\\{\}]
\PYG{g+gp}{\PYGZdl{} }sudo dnf update
\PYG{g+gp}{\PYGZdl{} }sudo dnf install git\PYGZhy{}all
\end{sphinxVerbatim}


\section{Initial configuration}
\label{\detokenize{gitinminutes:initial-configuration}}
\sphinxAtStartPar
Git ships with a tool called \sphinxcode{\sphinxupquote{git config}} that allows you to set multiple configuration variables. These variables control how Git looks and behaves.

\sphinxAtStartPar
Depending on your system, the configuration variables will be stored at different locations. For further details about the topic, check Git’s official documentation: \sphinxhref{https://git-scm.com/book/en/v2/Getting-Started-First-Time-Git-Setup}{Getting Started \sphinxhyphen{} First\sphinxhyphen{}Time Git Setup}%
\begin{footnote}[7]\sphinxAtStartFootnote
\sphinxnolinkurl{https://git-scm.com/book/en/v2/Getting-Started-First-Time-Git-Setup}
%
\end{footnote}.

\sphinxAtStartPar
Once you have installed Git, you should set your credentials by indicating your \sphinxstylestrong{user name} and \sphinxstylestrong{email}. To do so, type the following commands:

\begin{sphinxVerbatim}[commandchars=\\\{\}]
\PYG{g+gp}{\PYGZdl{} }git config \PYGZhy{}\PYGZhy{}global user.name \PYG{l+s+s2}{\PYGZdq{}Random User\PYGZdq{}}
\PYG{g+gp}{\PYGZdl{} }git config \PYGZhy{}\PYGZhy{}global user.email randomuser@test.com
\end{sphinxVerbatim}

\sphinxAtStartPar
To check all your personal settings, type the following command:

\begin{sphinxVerbatim}[commandchars=\\\{\}]
\PYG{g+gp}{\PYGZdl{}  }git config \PYGZhy{}\PYGZhy{}list
\end{sphinxVerbatim}


\chapter{Git essential commands}
\label{\detokenize{gitinminutes:git-essential-commands}}
\sphinxAtStartPar
Here are the most essential commands that will get you up and running within minutes.


\section{Initializing a new repository}
\label{\detokenize{gitinminutes:initializing-a-new-repository}}
\sphinxAtStartPar
If you already have a project, you can immediately navigate to the relevant folder, then initialize an empty repository with the command:

\begin{sphinxVerbatim}[commandchars=\\\{\}]
\PYG{g+gp}{\PYGZdl{} }git init
\end{sphinxVerbatim}


\section{Cloning an existing repository}
\label{\detokenize{gitinminutes:cloning-an-existing-repository}}
\sphinxAtStartPar
To clone an existing repository, type the command:

\begin{sphinxVerbatim}[commandchars=\\\{\}]
\PYG{g+gp}{\PYGZdl{} }git clone \PYGZlt{}URL\PYGZgt{}
\end{sphinxVerbatim}

\sphinxAtStartPar
For instance, if we want to clone the documentation repository from the collaboration platform \sphinxstyleemphasis{Codeberg}, then we will type the following command:

\begin{sphinxVerbatim}[commandchars=\\\{\}]
\PYG{g+gp}{\PYGZdl{} }git clone https://codeberg.org/Codeberg/Documentation.git
\end{sphinxVerbatim}


\section{Adding files}
\label{\detokenize{gitinminutes:adding-files}}
\sphinxAtStartPar
Git will not begin tracking your files unless you add them. To add all the files that are available in your directory to Git, type the command:

\begin{sphinxVerbatim}[commandchars=\\\{\}]
\PYG{g+gp}{\PYGZdl{} }git add \PYGZhy{}A
\end{sphinxVerbatim}

\sphinxAtStartPar
You can achieve the same result with the following command:

\begin{sphinxVerbatim}[commandchars=\\\{\}]
\PYG{g+gp}{\PYGZdl{} }git add .
\end{sphinxVerbatim}

\sphinxAtStartPar
Either way, the files existing in your project’s folder will be added recursively to Git’s index.

\sphinxAtStartPar
To add a single file called ‘foo’, type the command:

\begin{sphinxVerbatim}[commandchars=\\\{\}]
\PYG{g+gp}{\PYGZdl{} }git add foo
\end{sphinxVerbatim}


\section{Committing changes}
\label{\detokenize{gitinminutes:committing-changes}}
\sphinxAtStartPar
To commit your changes with a message, type the command:

\begin{sphinxVerbatim}[commandchars=\\\{\}]
\PYG{g+gp}{\PYGZdl{} }git commit \PYGZhy{}m \PYG{l+s+s2}{\PYGZdq{}Initial commit for Git\PYGZsq{}s documentation project\PYGZdq{}}
\end{sphinxVerbatim}

\begin{sphinxadmonition}{note}{Note:}
\sphinxAtStartPar
If you do not insert a commit message at the time of committing your files, i.e. if you only type \sphinxcode{\sphinxupquote{git commit}}, Git will launch the default text editor that is set in your environment variables.
\end{sphinxadmonition}


\section{Checking the status}
\label{\detokenize{gitinminutes:checking-the-status}}
\sphinxAtStartPar
If you want to check the status of the project’s files, type the command:

\begin{sphinxVerbatim}[commandchars=\\\{\}]
\PYG{g+gp}{\PYGZdl{} }git status
\end{sphinxVerbatim}

\sphinxAtStartPar
You will then get something like this:

\begin{sphinxVerbatim}[commandchars=\\\{\}]
\PYG{g+go}{ On branch maindoc}
\PYG{g+go}{ Changes not staged for commit:}
\PYG{g+gp+gpVirtualEnv}{(use \PYGZdq{}git add \PYGZlt{}file\PYGZgt{}...\PYGZdq{} to update what will be committed)}
\PYG{g+gp+gpVirtualEnv}{(use \PYGZdq{}git restore \PYGZlt{}file\PYGZgt{}...\PYGZdq{} to discard changes in working directory)}
\PYG{g+go}{ modified:   build/doctrees/environment.pickle}
\PYG{g+go}{ modified:   build/doctrees/gitcommands.doctree}
\PYG{g+go}{ modified:   build/doctrees/index.doctree}
\PYG{g+go}{ modified:   build/html/\PYGZus{}sources/gitcommands.rst.txt}
\PYG{g+go}{ modified:   build/html/\PYGZus{}static/pygments.css}
\PYG{g+go}{ modified:   build/html/gitcommands.html}
\PYG{g+go}{ modified:   build/html/index.html}
\PYG{g+go}{ modified:   build/html/objects.inv}
\PYG{g+go}{ modified:   build/html/searchindex.js}
\PYG{g+go}{ modified:   source/conf.py}
\PYG{g+go}{ modified:   source/gitcommands.rst}
\end{sphinxVerbatim}

\sphinxAtStartPar
The command \sphinxcode{\sphinxupquote{git status}} provides the default description. To get a verbose description, type the following command:

\begin{sphinxVerbatim}[commandchars=\\\{\}]
\PYG{g+gp}{\PYGZdl{} }git status \PYGZhy{}v
\end{sphinxVerbatim}

\sphinxAtStartPar
If you prefer a shorter description, type the command:

\begin{sphinxVerbatim}[commandchars=\\\{\}]
\PYG{g+gp}{\PYGZdl{} }git status \PYGZhy{}s
\end{sphinxVerbatim}

\sphinxAtStartPar
This will you give you the following result:

\begin{sphinxVerbatim}[commandchars=\\\{\}]
\PYG{g+go}{M build/doctrees/environment.pickle}
\PYG{g+go}{M build/doctrees/gitcommands.doctree}
\PYG{g+go}{M build/doctrees/index.doctree}
\PYG{g+go}{M build/html/\PYGZus{}sources/gitcommands.rst.txt}
\PYG{g+go}{M build/html/\PYGZus{}static/pygments.css}
\PYG{g+go}{M build/html/gitcommands.html}
\PYG{g+go}{M build/html/index.html}
\PYG{g+go}{M build/html/objects.inv}
\PYG{g+go}{M build/html/searchindex.js}
\PYG{g+go}{M source/conf.py}
\PYG{g+go}{M source/gitcommands.rst}
\end{sphinxVerbatim}

\sphinxAtStartPar
In the example above, The letter \sphinxstylestrong{M} at the beginning of each line means \sphinxcode{\sphinxupquote{Modified}}.


\section{Comparing with ‘diff’}
\label{\detokenize{gitinminutes:comparing-with-diff}}
\sphinxAtStartPar
To compare your local index with the repository, type the following command:

\begin{sphinxVerbatim}[commandchars=\\\{\}]
\PYG{g+gp}{\PYGZdl{} }git diff
\end{sphinxVerbatim}

\sphinxAtStartPar
You will then get a result similar to this:

\begin{sphinxVerbatim}[commandchars=\\\{\}]
\PYG{g+go}{diff \PYGZhy{}\PYGZhy{}git a/docs/build/doctrees/environment.pickle b/docs/build/doctrees/environment.pickle}
\PYG{g+go}{index 76e71d8..ca8948d 100644}
\PYG{g+go}{Binary files a/docs/build/doctrees/environment.pickle and b/docs/build/doctrees/environment.pickle differ}
\PYG{g+go}{diff \PYGZhy{}\PYGZhy{}git a/docs/build/doctrees/gitcommands.doctree b/docs/build/doctrees/gitcommands.doctree}
\PYG{g+go}{index b4e2fe0..5821717 100644}
\PYG{g+go}{Binary files a/docs/build/doctrees/gitcommands.doctree and b/docs/build/doctrees/gitcommands.doctree differ}
\PYG{g+go}{diff \PYGZhy{}\PYGZhy{}git a/docs/build/doctrees/index.doctree b/docs/build/doctrees/index.doctree}
\PYG{g+go}{index dc2937d..d476ecb 100644}
\PYG{g+go}{Binary files a/docs/build/doctrees/index.doctree and b/docs/build/doctrees/index.doctree differ}
\PYG{g+go}{diff \PYGZhy{}\PYGZhy{}git a/docs/build/html/\PYGZus{}sources/gitcommands.rst.txt b/docs/build/html/\PYGZus{}sources/gitcommands.rst.txt}
\PYG{g+go}{index 9a17fde..962687d 100644}
\PYG{g+go}{\PYGZhy{}\PYGZhy{}\PYGZhy{} a/docs/build/html/\PYGZus{}sources/gitcommands.rst.txt}
\PYG{g+go}{+++ b/docs/build/html/\PYGZus{}sources/gitcommands.rst.txt}
\PYG{g+go}{@@ \PYGZhy{}24,8 +24,8 @@ In a Git workflow, your files will basically go through 3 diff}
\end{sphinxVerbatim}

\sphinxAtStartPar
If you want the same result in a table format, add the option \sphinxcode{\sphinxupquote{\sphinxhyphen{}\sphinxhyphen{}stat}} to the initial command \sphinxcode{\sphinxupquote{git status}}:

\begin{sphinxVerbatim}[commandchars=\\\{\}]
\PYG{g+gp}{\PYGZdl{} }git diff \PYGZhy{}\PYGZhy{}stat
\end{sphinxVerbatim}

\sphinxAtStartPar
The command above will display something similar to this:

\begin{sphinxVerbatim}[commandchars=\\\{\}]
\PYG{g+go}{docs/build/doctrees/environment.pickle        | Bin 15477 \PYGZhy{}\PYGZgt{} 15570 bytes}
\PYG{g+go}{docs/build/doctrees/gitcommands.doctree      | Bin 14576 \PYGZhy{}\PYGZgt{} 20749 bytes}
\PYG{g+go}{docs/build/doctrees/index.doctree             | Bin 9193 \PYGZhy{}\PYGZgt{} 8977 bytes}
\PYG{g+go}{docs/build/html/\PYGZus{}sources/gitcommands.rst.txt |  78 ++++++++++++++++++++++\PYGZhy{}}
\PYG{g+go}{docs/build/html/\PYGZus{}static/pygments.css          |   6 +\PYGZhy{}}
\PYG{g+go}{docs/build/html/gitcommands.html             |  86 +++++++++++++++++++++\PYGZhy{}\PYGZhy{}\PYGZhy{}\PYGZhy{}\PYGZhy{}}
\PYG{g+go}{docs/build/html/index.html                    |   9 ++\PYGZhy{}}
\PYG{g+go}{docs/build/html/objects.inv                   | Bin 402 \PYGZhy{}\PYGZgt{} 414 bytes}
\PYG{g+go}{docs/build/html/searchindex.js                |   2 +\PYGZhy{}}
\PYG{g+go}{docs/source/conf.py                           |   2 +\PYGZhy{}}
\PYG{g+go}{docs/source/gitcommands.rst                  |  78 ++++++++++++++++++++++\PYGZhy{}}
\PYG{g+go}{11 files changed, 228 insertions(+), 33 deletions(\PYGZhy{})}
\end{sphinxVerbatim}


\section{Managing branches}
\label{\detokenize{gitinminutes:managing-branches}}
\sphinxAtStartPar
At the beginning of each project, you will have a \sphinxcode{\sphinxupquote{master branch}}, also called \sphinxcode{\sphinxupquote{main branch}} in newer terminology.

\sphinxAtStartPar
To view all current branches, type the following command:

\begin{sphinxVerbatim}[commandchars=\\\{\}]
\PYGZdl{} git branch \PYGZhy{}a
\end{sphinxVerbatim}

\newpage

\sphinxAtStartPar
If you want to create a new branch and switch to it, type the command:

\begin{sphinxVerbatim}[commandchars=\\\{\}]
\PYGZdl{} git checkout \PYGZhy{}b \PYGZlt{}new\PYGZhy{}branch\PYGZgt{}
\end{sphinxVerbatim}

\begin{sphinxadmonition}{note}{Note:}
\sphinxAtStartPar
The Git command \sphinxcode{\sphinxupquote{checkout}} allows you to switch to a different branch, which then becomes the \sphinxcode{\sphinxupquote{HEAD}} branch. \sphinxcode{\sphinxupquote{HEAD}} is a special pointer that points to the branch you are currently on.
\end{sphinxadmonition}


\section{Deleting branches}
\label{\detokenize{gitinminutes:deleting-branches}}
\sphinxAtStartPar
Each time you want to introduce a fix or a feature, you create a new dedicated branch to separate your work from the codebase. Once you deploy your contribution, you can delete that particular branch both locally and remotely. Note that the local and the remote branch are two completely different objects in Git, i.e. deleting a branch locally does not mean that its remote counterpart will be removed, and vice versa.


\subsection{Deleting branches locally}
\label{\detokenize{gitinminutes:deleting-branches-locally}}
\sphinxAtStartPar
You can delete a local branch with the \sphinxcode{\sphinxupquote{\sphinxhyphen{}d}} option. Since you cannot delete the branch you are currently on, you will first have to checkout a different branch:

\begin{sphinxVerbatim}[commandchars=\\\{\}]
\PYG{n}{git} \PYG{n}{checkout} \PYG{o}{\PYGZlt{}}\PYG{o+ow}{not}\PYG{o}{\PYGZhy{}}\PYG{n}{to}\PYG{o}{\PYGZhy{}}\PYG{n}{be}\PYG{o}{\PYGZhy{}}\PYG{n}{deleted} \PYG{n}{branch}\PYG{o}{\PYGZgt{}}

\PYG{n}{git} \PYG{n}{branch} \PYG{o}{\PYGZhy{}}\PYG{n}{d} \PYG{o}{\PYGZlt{}}\PYG{n}{branch}\PYG{o}{\PYGZhy{}}\PYG{n}{to}\PYG{o}{\PYGZhy{}}\PYG{n}{delete}\PYG{o}{\PYGZgt{}}
\end{sphinxVerbatim}

\begin{sphinxadmonition}{note}{Note:}
\sphinxAtStartPar
The \sphinxcode{\sphinxupquote{\sphinxhyphen{}d}} option only allows you to delete branches that have already been pushed and merged with their respective remote branches.
\end{sphinxadmonition}

\sphinxAtStartPar
To force local deletion \sphinxstylestrong{without} prior push and merge, use the \sphinxcode{\sphinxupquote{\sphinxhyphen{}D}} option:

\begin{sphinxVerbatim}[commandchars=\\\{\}]
\PYG{n}{git} \PYG{n}{branch} \PYG{o}{\PYGZhy{}}\PYG{n}{D} \PYG{o}{\PYGZlt{}}\PYG{n}{branch}\PYG{o}{\PYGZhy{}}\PYG{n}{to}\PYG{o}{\PYGZhy{}}\PYG{n}{delete}\PYG{o}{\PYGZgt{}}
\end{sphinxVerbatim}


\subsection{Deleting branches remotely}
\label{\detokenize{gitinminutes:deleting-branches-remotely}}
\sphinxAtStartPar
To delete a branch on a remote repo, use the \sphinxcode{\sphinxupquote{git push}} command in combination with the \sphinxcode{\sphinxupquote{\sphinxhyphen{}\sphinxhyphen{}delete}} option as shown below:

\begin{sphinxVerbatim}[commandchars=\\\{\}]
\PYG{n}{git} \PYG{n}{push} \PYG{o}{\PYGZlt{}}\PYG{n}{remote}\PYG{o}{\PYGZgt{}} \PYG{o}{\PYGZhy{}}\PYG{o}{\PYGZhy{}}\PYG{n}{delete} \PYG{o}{\PYGZlt{}}\PYG{n}{branch}\PYG{o}{\PYGZgt{}}
\end{sphinxVerbatim}

\sphinxAtStartPar
Example:

\begin{sphinxVerbatim}[commandchars=\\\{\}]
\PYG{n}{git} \PYG{n}{push} \PYG{n}{origin} \PYG{o}{\PYGZhy{}}\PYG{o}{\PYGZhy{}}\PYG{n}{delete} \PYG{n}{feature}\PYG{o}{/}\PYG{n}{captcha}
\end{sphinxVerbatim}


\section{Forking from a repository}
\label{\detokenize{gitinminutes:forking-from-a-repository}}
\sphinxAtStartPar
\sphinxcode{\sphinxupquote{Forking}} is the process of creating a completely new copy of a public repository. Forking allows you to work on your own copy of the project before submitting your changes back to the main repository through a \sphinxcode{\sphinxupquote{pull request}}.


\section{Managing remotes}
\label{\detokenize{gitinminutes:managing-remotes}}
\sphinxAtStartPar
Managing your remotes, i.e. remote servers, involves verifying the available remotes, setting a particular remote and removing references to remote branches, among other things.

\sphinxAtStartPar
To set a remote repository, type the command:

\begin{sphinxVerbatim}[commandchars=\\\{\}]
\PYGZdl{} git remote add origin \PYGZlt{}URL\PYGZgt{}
\end{sphinxVerbatim}

\begin{sphinxadmonition}{note}{Note:}
\sphinxAtStartPar
In the context of Git hosting platforms, \sphinxcode{\sphinxupquote{origin}} designates your own fork, while \sphinxcode{\sphinxupquote{upstream}} refers to the original repository that you have forked.
\end{sphinxadmonition}

\sphinxAtStartPar
To verify the remote repository, type the command:

\begin{sphinxVerbatim}[commandchars=\\\{\}]
\PYGZdl{} git remote \PYGZhy{}v
\end{sphinxVerbatim}

\sphinxAtStartPar
You will then get a result similar to this:

\begin{sphinxVerbatim}[commandchars=\\\{\}]
\PYG{n}{origin}   \PYG{n}{https}\PYG{p}{:}\PYG{o}{/}\PYG{o}{/}\PYG{n}{codeberg}\PYG{o}{.}\PYG{n}{org}\PYG{o}{/}\PYG{n}{Codeberg}\PYG{o}{/}\PYG{n}{Documentation}\PYG{o}{.}\PYG{n}{git} \PYG{p}{(}\PYG{n}{fetch}\PYG{p}{)}
\PYG{n}{origin}   \PYG{n}{https}\PYG{p}{:}\PYG{o}{/}\PYG{o}{/}\PYG{n}{codeberg}\PYG{o}{.}\PYG{n}{org}\PYG{o}{/}\PYG{n}{Codeberg}\PYG{o}{/}\PYG{n}{Documentation}\PYG{o}{.}\PYG{n}{git} \PYG{p}{(}\PYG{n}{push}\PYG{p}{)}
\end{sphinxVerbatim}

\sphinxAtStartPar
Note that the output contains 2 different terms at the end of each line, which are \sphinxcode{\sphinxupquote{fetch}} and \sphinxcode{\sphinxupquote{push}}: \sphinxcode{\sphinxupquote{fetch}} is the action of getting data from the remote repository, while \sphinxcode{\sphinxupquote{push}} means sending data to the remote.

\sphinxAtStartPar
To fetch data from your remote repository with its entire branches, run the command:

\begin{sphinxVerbatim}[commandchars=\\\{\}]
\PYGZdl{} git fetch \PYGZlt{}remote\PYGZgt{}
\end{sphinxVerbatim}

\sphinxAtStartPar
If you want to fetch a specific branch from the remote repository, run the following command:

\begin{sphinxVerbatim}[commandchars=\\\{\}]
\PYGZdl{} git fetch \PYGZlt{}remote\PYGZgt{} \PYGZlt{}branch\PYGZgt{}
\end{sphinxVerbatim}

\begin{sphinxadmonition}{attention}{Attention:}
\sphinxAtStartPar
The \sphinxcode{\sphinxupquote{fetch}} command allows you to download the data to your local repository, but it does \sphinxstylestrong{not} alter your local content. If you want to check out the fetched content, you will have to do it manually. Another possibility would be to use the \sphinxcode{\sphinxupquote{git pull}} command, which allows you to fetch the content from the remote server and merge it automatically into your local branch.
\end{sphinxadmonition}

\newpage

\sphinxAtStartPar
If you want to pull a single file from the remote repo, check the current remote repo with the command:

\begin{sphinxVerbatim}[commandchars=\\\{\}]
\PYGZdl{} git remote \PYGZhy{}v
\end{sphinxVerbatim}

\sphinxAtStartPar
Once you have confirmed that \sphinxcode{\sphinxupquote{origin}} is the name of your remote, run the following commands:

\begin{sphinxVerbatim}[commandchars=\\\{\}]
\PYGZdl{} git fetch \PYGZhy{}\PYGZhy{}all
\PYGZdl{} git checkout origin/main \PYGZhy{}\PYGZhy{} /path/to/your/file
\end{sphinxVerbatim}

\sphinxAtStartPar
To push your local commits to the remote repo, run the following command:

\begin{sphinxVerbatim}[commandchars=\\\{\}]
\PYGZdl{} git push \PYGZlt{}remote\PYGZgt{} \PYGZlt{}branch\PYGZgt{}
\end{sphinxVerbatim}

\sphinxAtStartPar
If a branch on your local fork is not synced with the latest commits from its remote counterpart, Git will not allow you to push your changes. This is to prevent you from rewriting the remote history that other contributors may be relying on. The \sphinxcode{\sphinxupquote{\sphinxhyphen{}\sphinxhyphen{}force}} option allows you to force the push in such cases and overwrite the history:

\begin{sphinxVerbatim}[commandchars=\\\{\}]
\PYGZdl{} git push \PYGZhy{}f \PYGZlt{}remote\PYGZgt{} \PYGZlt{}branch\PYGZgt{}
\end{sphinxVerbatim}

\sphinxAtStartPar
You can also achieve the same result by typing the following:

\begin{sphinxVerbatim}[commandchars=\\\{\}]
\PYG{n}{git} \PYG{n}{push} \PYG{o}{\PYGZlt{}}\PYG{n}{remote}\PYG{o}{\PYGZgt{}} \PYG{o}{\PYGZlt{}}\PYG{n}{branch}\PYG{o}{\PYGZgt{}} \PYG{o}{\PYGZhy{}}\PYG{o}{\PYGZhy{}}\PYG{n}{force}
\end{sphinxVerbatim}

\begin{sphinxadmonition}{attention}{Attention:}
\sphinxAtStartPar
Proceed with caution when using the \sphinxcode{\sphinxupquote{\sphinxhyphen{}\sphinxhyphen{}force}} option. Rewriting the commit history means that others cannot access the older commit history anymore. Here are some “safer” alternatives:
\begin{itemize}
\item {} 
\sphinxAtStartPar
Avoid force pushing commits on repos with a shared history.

\item {} 
\sphinxAtStartPar
Use \sphinxcode{\sphinxupquote{git revert}} to undo changes from existing commits.

\item {} 
\sphinxAtStartPar
Use the command \sphinxcode{\sphinxupquote{git push <remote> <branch> \sphinxhyphen{}\sphinxhyphen{}force\sphinxhyphen{}with\sphinxhyphen{}lease}}. This command will not rewrite any changes made by other team members on the remote repo.

\end{itemize}
\end{sphinxadmonition}

\sphinxAtStartPar
If you want to set a different repo, type the command:

\begin{sphinxVerbatim}[commandchars=\\\{\}]
\PYGZdl{} git remote set\PYGZhy{}url origin \PYGZlt{}URL\PYGZgt{}
\end{sphinxVerbatim}

\sphinxAtStartPar
In order to delete references to any remote branches that no longer exist, use the command:

\begin{sphinxVerbatim}[commandchars=\\\{\}]
\PYGZdl{} git remote prune origin
\end{sphinxVerbatim}


\section{Syncing your fork with upstream}
\label{\detokenize{gitinminutes:syncing-your-fork-with-upstream}}
\sphinxAtStartPar
If you have forked an upstream repo and started working on your local fork, you may notice after a while that your fork is out of sync with upstream. To remedy this situation and sync your fork with the upstream repo, run the following commands:

\begin{sphinxVerbatim}[commandchars=\\\{\}]
\PYGZdl{} git fetch upstream
\PYGZdl{} git checkout main
\PYGZdl{} git merge upstream/main
\end{sphinxVerbatim}

\begin{sphinxadmonition}{note}{Note:}
\sphinxAtStartPar
Use the term \sphinxcode{\sphinxupquote{main}} or \sphinxcode{\sphinxupquote{master}} in your commands according to the default terminology of your Git hosting platform, e.g. Codeberg or GitHub.
\end{sphinxadmonition}


\section{Viewing the commit history}
\label{\detokenize{gitinminutes:viewing-the-commit-history}}
\sphinxAtStartPar
During your project, you may want to go back to a “safe” commit if you encounter some issues at a certain point. There are other reasons why you might need access to the commit history, such as finding out \sphinxstyleemphasis{who} made \sphinxstyleemphasis{what} changes and \sphinxstyleemphasis{why}.

\sphinxAtStartPar
The \sphinxcode{\sphinxupquote{git log}} command allows you to track your project history in a reverse chronological order, i.e. the newest commit appears at the top.

\sphinxAtStartPar
Below is a sample output of the \sphinxcode{\sphinxupquote{git log}} command without any additional flags:

\begin{sphinxVerbatim}[commandchars=\\\{\}]
\PYGZdl{} git log
commit ad06d9ba80ba723b68b6600600e23bc85af7ff82 (HEAD \PYGZhy{}\PYGZgt{} easydocbranch, origin/easydocbranch)
Author: Faycal Alami\PYGZhy{}Hassani \PYGZlt{}anon@yme.com\PYGZgt{}
Date:   Thu Feb 17 21:43:38 2022 +0100

Updating content about metadata

commit 09ca7947a1935841ea4d76d3fe815ea988ad2c77
Author: Faycal Alami\PYGZhy{}Hassani \PYGZlt{}anon@yme.com\PYGZgt{}
Date:   Thu Feb 17 21:31:59 2022 +0100

Proofreading the Git article

commit b5ef042c0f907bfebb2c6917b5de1e072a3fd18a
Author: Faycal Alami\PYGZhy{}Hassani \PYGZlt{}anon@yme.com\PYGZgt{}
Date:   Thu Feb 17 20:29:28 2022 +0100

Finished proofreading the article
\end{sphinxVerbatim}

\sphinxAtStartPar
To get a compact overview of your commit history, you can combine the \sphinxcode{\sphinxupquote{git log}} command with the option \sphinxcode{\sphinxupquote{\sphinxhyphen{}\sphinxhyphen{}oneline}}. Each single line will then display the \sphinxstylestrong{commit ID} and the \sphinxstylestrong{first line} of the commit message, e.g.:

\begin{sphinxVerbatim}[commandchars=\\\{\}]
\PYGZdl{} git log \PYGZhy{}\PYGZhy{}oneline

fd9e2e4 Updating the table about HTTP verbs
91137e4 Adding information about HTTP methods and URIs
3b5f0e8 Adding content about FTP commands
41f2a36 Updating the article about Git
\end{sphinxVerbatim}

\begin{sphinxadmonition}{note}{Note:}
\sphinxAtStartPar
To get the greatest benefit from your commit history, always follow these two rules:
\begin{enumerate}
\sphinxsetlistlabels{\arabic}{enumi}{enumii}{}{.}%
\item {} 
\sphinxAtStartPar
Keep your commits as small as possible, i.e. each commit should include the smallest possible amount of changes. This ensures a logical organization of your commits and makes it easier to revert single changes.

\item {} 
\sphinxAtStartPar
Provide a good description in your commit message. The commit message should explain precisely what the commit does.

\end{enumerate}
\end{sphinxadmonition}


\section{Rebasing commits}
\label{\detokenize{gitinminutes:rebasing-commits}}
\sphinxAtStartPar
Git provides two mechanisms to integrate changes from one branch into another: \sphinxcode{\sphinxupquote{merge}} and \sphinxcode{\sphinxupquote{rebase}}.

\sphinxAtStartPar
The merge option is a \sphinxstyleemphasis{non\sphinxhyphen{}destructive} operation. It allows you to join two or more sequences of commits together “without” altering the project history.

\sphinxAtStartPar
Unlike the merge option, a rebase “rewrites” the project history by \sphinxstylestrong{reapplying} all the commits of a given branch on top of the base branch. With \sphinxcode{\sphinxupquote{git rebase}}, you can move an entire feature branch to place it on the tip of the base branch in the tree.

\begin{figure}[H]
\centering
\capstart

\noindent\sphinxincludegraphics[width=800\sphinxpxdimen,height=544\sphinxpxdimen]{{git-rebase}.pdf}
\caption{Rebasing commits – Picture under a \sphinxhref{https://creativecommons.org/publicdomain/zero/1.0/}{CC0 1.0 Universal}\sphinxfootnotemark[8] license}\label{\detokenize{gitinminutes:id2}}\end{figure}
%
\begin{footnotetext}[8]\sphinxAtStartFootnote
\sphinxnolinkurl{https://creativecommons.org/publicdomain/zero/1.0/}
%
\end{footnotetext}\ignorespaces 
\sphinxAtStartPar
To rebase a feature branch onto the main branch, run the following commands:

\begin{sphinxVerbatim}[commandchars=\\\{\}]
\PYG{n}{git} \PYG{n}{checkout} \PYG{o}{\PYGZlt{}}\PYG{n}{feature}\PYG{o}{\PYGZhy{}}\PYG{n}{branch}\PYG{o}{\PYGZgt{}}

\PYG{n}{git} \PYG{n}{rebase} \PYG{n}{main}
\end{sphinxVerbatim}

\begin{sphinxadmonition}{attention}{Attention:}
\sphinxAtStartPar
Rebasing alters the project history. This means you can rebase any local commits that you have not pushed yet, but you should \sphinxstylestrong{never} rebase shared commits on remote repositories. Otherwise, you risk altering the development history that other contributors may rely on.
\end{sphinxadmonition}


\section{Squashing commits}
\label{\detokenize{gitinminutes:squashing-commits}}
\sphinxAtStartPar
Squashing is the act of merging multiple commits into a single one. You can squash commits at any time with the \sphinxstyleemphasis{interactive rebase} feature.
For instance, to display the three latest commits, we will type the following command:

\begin{sphinxVerbatim}[commandchars=\\\{\}]
\PYG{n}{git} \PYG{n}{rebase} \PYG{o}{\PYGZhy{}}\PYG{n}{i} \PYG{n}{HEAD}\PYG{o}{\PYGZti{}}\PYG{l+m+mi}{3}
\end{sphinxVerbatim}

\begin{sphinxadmonition}{note}{Note:}
\sphinxAtStartPar
In the command above, the \sphinxcode{\sphinxupquote{n}} within \sphinxcode{\sphinxupquote{HEAD\textasciitilde{}n}} denotes the number of commits you want to go back. In this particular case, the HEAD branch will move three positions back to a previous commit.
\end{sphinxadmonition}

\sphinxAtStartPar
You should then get an output similar to this:

\begin{sphinxVerbatim}[commandchars=\\\{\}]
\PYG{n}{pick} \PYG{l+m+mi}{09}\PYG{n}{ca794} \PYG{n}{Proofreading} \PYG{n}{the} \PYG{n}{git} \PYG{n}{article}
\PYG{n}{pick} \PYG{n}{ad06d9b} \PYG{n}{Updating} \PYG{n}{content} \PYG{n}{about} \PYG{n}{metadata}
\PYG{n}{pick} \PYG{n}{b60f293} \PYG{n}{Introducing} \PYG{n}{changes} \PYG{n}{to} \PYG{n}{produce} \PYG{n}{PDF} \PYG{k}{with} \PYG{n}{LaTeX} \PYG{o+ow}{and} \PYG{n}{updating} \PYG{n}{article}

\PYG{c+c1}{\PYGZsh{} Rebase b5ef042..b60f293 onto b5ef042 (3 commands)}
\PYG{c+c1}{\PYGZsh{}}
\PYG{c+c1}{\PYGZsh{} Commands:}
\PYG{c+c1}{\PYGZsh{} p, pick \PYGZlt{}commit\PYGZgt{} = use commit}
\PYG{c+c1}{\PYGZsh{} r, reword \PYGZlt{}commit\PYGZgt{} = use commit, but edit the commit message}
\PYG{c+c1}{\PYGZsh{} e, edit \PYGZlt{}commit\PYGZgt{} = use commit, but stop for amending}
\PYG{c+c1}{\PYGZsh{} s, squash \PYGZlt{}commit\PYGZgt{} = use commit, but meld into previous commit}
\PYG{c+c1}{\PYGZsh{} f, fixup [\PYGZhy{}C | \PYGZhy{}c] \PYGZlt{}commit\PYGZgt{} = like \PYGZdq{}squash\PYGZdq{} but keep only the previous}
\PYG{c+c1}{\PYGZsh{}                    commit\PYGZsq{}s log message, unless \PYGZhy{}C is used, in which case}
\PYG{c+c1}{\PYGZsh{}                    keep only this commit\PYGZsq{}s message; \PYGZhy{}c is same as \PYGZhy{}C but}
\PYG{c+c1}{\PYGZsh{}                    opens the editor}
\PYG{c+c1}{\PYGZsh{} x, exec \PYGZlt{}command\PYGZgt{} = run command (the rest of the line) using shell}
\PYG{c+c1}{\PYGZsh{} b, break = stop here (continue rebase later with \PYGZsq{}git rebase \PYGZhy{}\PYGZhy{}continue\PYGZsq{})}
\PYG{c+c1}{\PYGZsh{} d, drop \PYGZlt{}commit\PYGZgt{} = remove commit}
\PYG{c+c1}{\PYGZsh{} l, label \PYGZlt{}label\PYGZgt{} = label current HEAD with a name}
\PYG{c+c1}{\PYGZsh{} t, reset \PYGZlt{}label\PYGZgt{} = reset HEAD to a label}
\end{sphinxVerbatim}

\newpage

\sphinxAtStartPar
If you replace \sphinxstylestrong{pick} by \sphinxstylestrong{squash} in one of the lines above, the line in question will be combined with the one above, e.g.:

\begin{sphinxVerbatim}[commandchars=\\\{\}]
\PYG{n}{pick} \PYG{l+m+mi}{09}\PYG{n}{ca794} \PYG{n}{Proofreading} \PYG{n}{the} \PYG{n}{git} \PYG{n}{article}
\PYG{n}{squash} \PYG{n}{ad06d9b} \PYG{n}{Updating} \PYG{n}{content} \PYG{n}{about} \PYG{n}{metadata}
\PYG{n}{squash} \PYG{n}{b60f293} \PYG{n}{Introducing} \PYG{n}{changes} \PYG{n}{to} \PYG{n}{produce} \PYG{n}{PDF} \PYG{k}{with} \PYG{n}{LaTeX} \PYG{o+ow}{and} \PYG{n}{updating} \PYG{n}{article}
\end{sphinxVerbatim}

\sphinxAtStartPar
Once you edit the commit message for the new compact commit, the interactive rebase will complete successfully. You should now have a single commit instead of three.


\section{Submitting separate pull requests}
\label{\detokenize{gitinminutes:submitting-separate-pull-requests}}
\sphinxAtStartPar
You may want to submit a separate pull request for each commit. To do so, you first have to reset your \sphinxcode{\sphinxupquote{main}} repo to sync it with \sphinxcode{\sphinxupquote{upstream}}:

\begin{sphinxVerbatim}[commandchars=\\\{\}]
\PYG{n}{git} \PYG{n}{checkout} \PYG{n}{main}
\PYG{n}{git} \PYG{n}{reset} \PYG{o}{\PYGZhy{}}\PYG{o}{\PYGZhy{}}\PYG{n}{hard} \PYG{n}{upstream}\PYG{o}{/}\PYG{n}{main}
\end{sphinxVerbatim}

\sphinxAtStartPar
The next step consists in creating a new branch for each new commit, then “cherry\sphinxhyphen{}picking” the relevant commit. The \sphinxcode{\sphinxupquote{git cherry\sphinxhyphen{}pick}} command allows you to re\sphinxhyphen{}apply the changes from a previous commit to the current active branch:

\begin{sphinxVerbatim}[commandchars=\\\{\}]
\PYG{n}{git} \PYG{n}{checkout} \PYG{o}{\PYGZhy{}}\PYG{n}{b} \PYG{n}{new}\PYG{o}{\PYGZhy{}}\PYG{n}{branch}
\PYG{n}{git} \PYG{n}{cherry}\PYG{o}{\PYGZhy{}}\PYG{n}{pick} \PYG{l+m+mf}{91137e4}
\PYG{n}{git} \PYG{n}{push} \PYG{o}{\PYGZhy{}}\PYG{o}{\PYGZhy{}}\PYG{n+nb}{set}\PYG{o}{\PYGZhy{}}\PYG{n}{upstream} \PYG{n}{origin} \PYG{n}{new}\PYG{o}{\PYGZhy{}}\PYG{n}{branch}
\end{sphinxVerbatim}

\sphinxstepscope


\chapter{Compression \& metadata removal}
\label{\detokenize{metadata-compression:compression-metadata-removal}}\label{\detokenize{metadata-compression::doc}}
\begin{figure}[H]
\centering
\capstart

\noindent\sphinxincludegraphics[scale=0.9]{{Data-Mike-Haynes}.jpeg}
\caption{Picture by Mike Haynes under \sphinxhref{https://creativecommons.org/publicdomain/zero/1.0/}{CC0 1.0 License}\sphinxfootnotemark[9]}\label{\detokenize{metadata-compression:id1}}\end{figure}
%
\begin{footnotetext}[9]\sphinxAtStartFootnote
\sphinxnolinkurl{https://creativecommons.org/publicdomain/zero/1.0/}
%
\end{footnotetext}\ignorespaces 

\section{Compressing from the command line}
\label{\detokenize{metadata-compression:compressing-from-the-command-line}}
\sphinxAtStartPar
There are multiple scenarios where you would need to compress your files, whether it is for a web development project or to send some attachments by email, just to name a few examples. You do not necessarily need to rely on proprietary software products or \sphinxstyleabbreviation{GUIs} (Graphical User Interfaces) to achieve these tasks, especially if your are a Linux user.

\sphinxAtStartPar
In fact, Linux has many command line tools that allow you to compress your PDF and image files easily. My favorite open source tools for the compression of PDF and image files are \sphinxcode{\sphinxupquote{ps2pdf}} and \sphinxcode{\sphinxupquote{jpegoptim}}, respectively.


\section{Compressing PDF with ps2pdf}
\label{\detokenize{metadata-compression:compressing-pdf-with-ps2pdf}}
\sphinxAtStartPar
\sphinxcode{\sphinxupquote{ps2pdf}} is a PostScript\sphinxhyphen{}to\sphinxhyphen{}PDF converter that uses \sphinxcode{\sphinxupquote{ghostscript}} to convert a PDF into a PostScript file before converting it back again. This process allows you to compress your initial PDF file. According to its man page, \sphinxcode{\sphinxupquote{ps2pdf}} provides nearly all the features that you would find in Adobe’s Acrobat ® product, Distiller ®.


\section{Installing ps2pdf on Linux}
\label{\detokenize{metadata-compression:installing-ps2pdf-on-linux}}
\sphinxAtStartPar
To install \sphinxcode{\sphinxupquote{ps2pdf}} with all the required dependencies, you need to install \sphinxcode{\sphinxupquote{ghostscript}}.

\sphinxAtStartPar
On Debian based distros, run the following commands:

\begin{sphinxVerbatim}[commandchars=\\\{\}]
\PYG{g+gp}{\PYGZdl{} }sudo apt\PYGZhy{}get update
\PYG{g+gp}{\PYGZdl{} }sudo apt install ghostscript
\end{sphinxVerbatim}

\sphinxAtStartPar
For Red Hat based distros, use the following commands:

\begin{sphinxVerbatim}[commandchars=\\\{\}]
\PYG{g+gp}{\PYGZdl{} }sudo dnf update
\PYG{g+gp}{\PYGZdl{} }sudo dnf install ghostscript
\end{sphinxVerbatim}


\section{ps2pdf commands}
\label{\detokenize{metadata-compression:ps2pdf-commands}}
\sphinxAtStartPar
To compress a file without any additional options, type the following command:

\begin{sphinxVerbatim}[commandchars=\\\{\}]
\PYG{g+gp}{\PYGZdl{} }ps2pdf  \PYG{o}{[}options...\PYG{o}{]} \PYG{o}{\PYGZob{}}input.\PYG{o}{[}e\PYG{o}{]}ps\PYG{p}{|}\PYGZhy{}\PYG{o}{\PYGZcb{}} \PYG{o}{[}output.pdf\PYG{p}{|}\PYGZhy{}\PYG{o}{]}
\end{sphinxVerbatim}

\sphinxAtStartPar
Note that \sphinxcode{\sphinxupquote{ps2pdf}} uses the same options as \sphinxcode{\sphinxupquote{ghostscript}}.

\sphinxAtStartPar
Depending on your work scenario, you can achieve the best results in terms of file compression and image quality by using the option \sphinxcode{\sphinxupquote{\sphinxhyphen{}dPDFSETTINGS=/ebook}}:

\begin{sphinxVerbatim}[commandchars=\\\{\}]
\PYG{g+gp}{\PYGZdl{} }ps2pdf \PYGZhy{}dPDFSETTINGS\PYG{o}{=}/ebook input.pdf output.pdf
\end{sphinxVerbatim}


\chapter{Metadata and privacy implications}
\label{\detokenize{metadata-compression:metadata-and-privacy-implications}}
\sphinxAtStartPar
Metadata reveal more about you than you might imagine. Here is an example of the metadata that were extracted from an image file taken by a conventional smartphone: \sphinxcode{\sphinxupquote{Image Type}}, \sphinxcode{\sphinxupquote{Width}}, \sphinxcode{\sphinxupquote{Height}}, \sphinxcode{\sphinxupquote{Exposure Time}}, \sphinxcode{\sphinxupquote{Aperture Value}}, \sphinxcode{\sphinxupquote{ISO Speed Rating}}, \sphinxcode{\sphinxupquote{Flash Fired}}, \sphinxcode{\sphinxupquote{Metering Mode}}, \sphinxcode{\sphinxupquote{Exposure Program}}, \sphinxcode{\sphinxupquote{Focal Length}}, \sphinxcode{\sphinxupquote{Software}}, \sphinxcode{\sphinxupquote{Camera Brand}}, \sphinxcode{\sphinxupquote{Camera Model}}, and \sphinxcode{\sphinxupquote{Date Taken}}.

\sphinxAtStartPar
With this information at hand, malicious users can easily search for the most recent vulnerabilities associated with your device and craft a custom payload. Therefore, it is always good practice to remove metadata from your files before handing them over.


\section{Installing mat2 on Linux}
\label{\detokenize{metadata-compression:installing-mat2-on-linux}}
\sphinxAtStartPar
\sphinxcode{\sphinxupquote{mat2}} is a metadata anonymization toolkit that runs from the command line. \sphinxcode{\sphinxupquote{mat2}} allows you to remove metadata from a wide range of file formats, including archive, image, office, audio, video and PDF files.

\sphinxAtStartPar
To install \sphinxcode{\sphinxupquote{mat2}} on Debian based distros, run the following commands:

\begin{sphinxVerbatim}[commandchars=\\\{\}]
\PYG{g+gp}{\PYGZdl{} }sudo apt\PYGZhy{}get update
\PYG{g+gp}{\PYGZdl{} }sudo apt install mat2
\end{sphinxVerbatim}

\sphinxAtStartPar
For Red Hat based distros, use the following commands:

\begin{sphinxVerbatim}[commandchars=\\\{\}]
\PYG{g+gp}{\PYGZdl{} }sudo dnf update
\PYG{g+gp}{\PYGZdl{} }sudo dnf install mat2
\end{sphinxVerbatim}


\section{Removing metadata with mat2}
\label{\detokenize{metadata-compression:removing-metadata-with-mat2}}
\sphinxAtStartPar
mat2 does not overwrite the source file. Instead, it will generate a new output file that contains the word \sphinxstyleemphasis{cleaned} between the filename and the file extension. If you run the command below, mat2 will generate a new file called \sphinxstyleemphasis{foo.cleaned.pdf}:

\begin{sphinxVerbatim}[commandchars=\\\{\}]
\PYG{g+gp}{\PYGZdl{} }mat2 foo.pdf
\end{sphinxVerbatim}


\chapter{PDF forensics and safety measures}
\label{\detokenize{metadata-compression:pdf-forensics-and-safety-measures}}
\sphinxAtStartPar
As a general rule of thumb, you should never, ever open PDF files in a productive environment, even if you receive such files from people you trust. The reason for this is pretty obvious, since the persons you trust may themselves not be aware of the presence of an embedded payload in the PDF file.

\sphinxAtStartPar
For PDF files that do not contain any sensitive information, you can run a check on \sphinxhref{https://www.virustotal.com/}{VirusTotal}%
\begin{footnote}[10]\sphinxAtStartFootnote
\sphinxnolinkurl{https://www.virustotal.com/}
%
\end{footnote}. Beware though, that hackers also run a preliminary test on VirusTotal to make sure that their malicious payloads will not be flagged.

\sphinxAtStartPar
For an in\sphinxhyphen{}depth analysis, it is recommended to use forensic tools such as \sphinxcode{\sphinxupquote{pdfid.py}} in combination with the PDF parser \sphinxcode{\sphinxupquote{pdf\sphinxhyphen{}parser.py}} from \sphinxhref{https://blog.didierstevens.com/programs/pdf-tools/}{Didier Stevens}%
\begin{footnote}[11]\sphinxAtStartFootnote
\sphinxnolinkurl{https://blog.didierstevens.com/programs/pdf-tools/}
%
\end{footnote}.

\begin{sphinxadmonition}{note}{Note:}
\sphinxAtStartPar
Even when using your tools of choice to analyze suspicious PDF files, you should always perform your analysis on a virtual environment or in a sandbox, with no connection to any other devices or a network. Remember, never run these tests on a productive environment!
\end{sphinxadmonition}

\sphinxAtStartPar
As a safety measure, check also if your PDF reader supports JavaScript by default and disable it. There are multiple open\sphinxhyphen{}source PDF readers that do not render JavaScript at all.

\sphinxstepscope


\chapter{MTU Size and Connectivity Issues}
\label{\detokenize{mtu-connectivity:mtu-size-and-connectivity-issues}}\label{\detokenize{mtu-connectivity::doc}}
\begin{figure}[H]
\centering
\capstart

\noindent\sphinxincludegraphics[scale=0.5]{{mtu-connectivity}.png}
\caption{Picture by kreatikar \sphinxhref{https://pixabay.com}{(pixabay.com)}\sphinxfootnotemark[12]}\label{\detokenize{mtu-connectivity:id2}}\end{figure}
%
\begin{footnotetext}[12]\sphinxAtStartFootnote
\sphinxnolinkurl{https://pixabay.com}
%
\end{footnotetext}\ignorespaces 

\section{Factors impacting network connectivity}
\label{\detokenize{mtu-connectivity:factors-impacting-network-connectivity}}
\sphinxAtStartPar
Making changes to your network configuration can sometimes result in connection inconsistencies, such as a partial or complete loss of connectivity. The reasons for such incidents are multiple and range from DNS misconfiguration to inappropriate firewall rules or hardware failures, among other things.

\sphinxAtStartPar
One additional aspect that you should consider when some sites become unreachable after changing the network configuration is the MTU size. MTU stands for \sphinxcode{\sphinxupquote{Maximum Transmission Unit}} and is a concept that relates to the \sphinxhref{https://www.redhat.com/sysadmin/osi-model-bean-dip}{OSI reference model}%
\begin{footnote}[13]\sphinxAtStartFootnote
\sphinxnolinkurl{https://www.redhat.com/sysadmin/osi-model-bean-dip}
%
\end{footnote}, \sphinxhref{https://www.linuxjunkies.org/network/tcpip/general-description-of-the-tcp-ip-protocols/}{IP datagram size}%
\begin{footnote}[14]\sphinxAtStartFootnote
\sphinxnolinkurl{https://www.linuxjunkies.org/network/tcpip/general-description-of-the-tcp-ip-protocols/}
%
\end{footnote} and \sphinxhref{https://packetpushers.net/ip-fragmentation-in-detail/}{IP fragmentation}%
\begin{footnote}[15]\sphinxAtStartFootnote
\sphinxnolinkurl{https://packetpushers.net/ip-fragmentation-in-detail/}
%
\end{footnote}.


\chapter{Analyzing network traffic}
\label{\detokenize{mtu-connectivity:analyzing-network-traffic}}
\sphinxAtStartPar
If you have a \sphinxhref{https://www.cisco.com/c/en/us/products/switches/what-is-a-managed-switch.html}{managed switch}%
\begin{footnote}[16]\sphinxAtStartFootnote
\sphinxnolinkurl{https://www.cisco.com/c/en/us/products/switches/what-is-a-managed-switch.html}
%
\end{footnote} and face connectivity issues after altering your network configuration, a recommended method to capture and analyze your traffic would be to use port mirroring in combination with packet/network analyzers such as \sphinxhref{https://www.linuxjournal.com/content/tcpdump-fu}{tcpdump}%
\begin{footnote}[17]\sphinxAtStartFootnote
\sphinxnolinkurl{https://www.linuxjournal.com/content/tcpdump-fu}
%
\end{footnote} and \sphinxhref{https://www.linuxjournal.com/content/tcp-analysis-wireshark}{wireshark}%
\begin{footnote}[18]\sphinxAtStartFootnote
\sphinxnolinkurl{https://www.linuxjournal.com/content/tcp-analysis-wireshark}
%
\end{footnote}.

\sphinxAtStartPar
Port mirroring is a technique that allows you to forward a copy of all inbound and outbound packets from a single port or an entire VLAN to a target port for diagnostic purposes. You can hence configure a \sphinxcode{\sphinxupquote{mirrored port}} for the traffic to be mirrored, and a \sphinxcode{\sphinxupquote{monitor port}} for your local traffic destination.

\begin{figure}[H]
\centering
\capstart

\noindent\sphinxincludegraphics[width=498.32424mm,height=328.00891mm]{{port-mirroring}.pdf}
\caption{A simple network configuration for port mirroring}\label{\detokenize{mtu-connectivity:id3}}\end{figure}


\chapter{Find out the MTU of your Ethernet interface}
\label{\detokenize{mtu-connectivity:find-out-the-mtu-of-your-ethernet-interface}}
\sphinxAtStartPar
In this particular case, it turned out that a connection timeout occured each time the client was trying to establish a connection with the destination server. This can happen when a gateway along the connection path is using an MTU smaller than the one we are using, and the IPv4 datagram is not allowed to be fragmented (\sphinxcode{\sphinxupquote{DF flag}} set to “1”).

\begin{figure}[H]
\centering
\capstart

\noindent\sphinxincludegraphics[scale=1.0]{{mtu-difference}.pdf}
\caption{MTU difference along the transmission path}\label{\detokenize{mtu-connectivity:id4}}\end{figure}

\sphinxAtStartPar
An IPv4 datagram consists of two parts: a \sphinxcode{\sphinxupquote{header}} and a \sphinxcode{\sphinxupquote{payload}}. The header contains fields that are essential for data transmission, while the payload encloses the actual data. The default MTU size for Ethernet is 1500 bytes.

\sphinxAtStartPar
On Linux machines, you can check the MTU size of your ethernet interface through the following command:

\begin{sphinxVerbatim}[commandchars=\\\{\}]
\PYG{g+gp}{\PYGZdl{} }ip a \PYG{p}{|} grep mtu
\end{sphinxVerbatim}

\begin{sphinxadmonition}{note}{Note:}
\sphinxAtStartPar
The \sphinxcode{\sphinxupquote{ip a}} command also allows you to list all the available interfaces on your machine with their corresponding IPs.
\end{sphinxadmonition}


\chapter{Change the MTU size permanently}
\label{\detokenize{mtu-connectivity:change-the-mtu-size-permanently}}
\sphinxAtStartPar
Fot the purpose of this guide, we are going to assume that the name of your ethernet interface is \sphinxcode{\sphinxupquote{eth0}}. Use the \sphinxcode{\sphinxupquote{ip a}} command to check the actual name of your ethernet interface.

\sphinxAtStartPar
To change the MTU size on Debian based distros, run the following command:

\begin{sphinxVerbatim}[commandchars=\\\{\}]
\PYG{g+gp}{\PYGZdl{} }sudo nano /etc/network/interfaces
\end{sphinxVerbatim}

\sphinxAtStartPar
Then set a lower MTU value (e.g. 1464) for the required interface by adding a corresponding line at the bottom:

\begin{sphinxVerbatim}[commandchars=\\\{\}]
\PYG{g+gp}{\PYGZdl{} }mtu \PYG{l+m}{1464}
\end{sphinxVerbatim}

\sphinxAtStartPar
Save and close the file, then restart the networking services by running the following command:

\begin{sphinxVerbatim}[commandchars=\\\{\}]
\PYG{g+gp}{\PYGZdl{} }sudo service networking restart
\end{sphinxVerbatim}

\sphinxAtStartPar
To change the MTU size on Red Hat based distros, run the following command:

\begin{sphinxVerbatim}[commandchars=\\\{\}]
\PYG{g+gp}{\PYGZdl{} }sudo nano /etc/sysconfig/network\PYGZhy{}scripts/ifcfg\PYGZhy{}eth0
\end{sphinxVerbatim}

\sphinxAtStartPar
Then set a lower MTU value (e.g. 1464) for the required interface by adding a corresponding line at the bottom:

\begin{sphinxVerbatim}[commandchars=\\\{\}]
\PYG{g+gp}{\PYGZdl{} }\PYG{n+nv}{MTU}\PYG{o}{=}\PYG{l+s+s2}{\PYGZdq{}1464\PYGZdq{}}
\end{sphinxVerbatim}

\sphinxAtStartPar
Save and close the file, then restart the networking services by running the following command:

\begin{sphinxVerbatim}[commandchars=\\\{\}]
\PYG{g+gp}{\PYGZdl{} }sudo service networking restart
\end{sphinxVerbatim}

\begin{sphinxadmonition}{note}{Note:}
\sphinxAtStartPar
The minimum allowed value for IPv6 is 1280. Moreover, IPv6 handles fragmentation in a completely different way to that of IPv4. For further information about the differences, check the article about \sphinxhref{https://packetpushers.net/ip-fragmentation-in-detail/}{IP fragmentation}%
\begin{footnote}[19]\sphinxAtStartFootnote
\sphinxnolinkurl{https://packetpushers.net/ip-fragmentation-in-detail/}
%
\end{footnote}.
\end{sphinxadmonition}



\renewcommand{\indexname}{Index}
\footnotesize\raggedright\printindex
\end{document}